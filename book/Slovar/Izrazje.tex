% !TeX spellcheck = de_DE
%%%%%%%%%%%%%%%%%%%%%%%%%%%%%%%%%%%%%%%%%
% Dictionary
% LaTeX Template
% Version 1.0 (20/12/14)
%
% This template has been downloaded from:
% http://www.LaTeXTemplates.com
%
% Original author:
% Vel (vel@latextemplates.com) inspired by a template by Marc Lavaud
%
% License:
% CC BY-NC-SA 3.0 (http://creativecommons.org/licenses/by-nc-sa/3.0/)
%
%%%%%%%%%%%%%%%%%%%%%%%%%%%%%%%%%%%%%%%%%

%----------------------------------------------------------------------------------------
%	PACKAGES AND OTHER DOCUMENT CONFIGURATIONS
%----------------------------------------------------------------------------------------

\documentclass[10pt,a4paper,twoside]{article} % 10pt font size, A4 paper and two-sided margins

\usepackage[top=3.5cm,bottom=3.5cm,left=3.7cm,right=4.7cm,columnsep=30pt]{geometry} % Document margins and spacings
\usepackage[slovene]{babel}
%\catcode`\"=13

\usepackage[utf8]{inputenc} % Required for inputting international characters
\usepackage[T1]{fontenc} % Output font encoding for international characters

\usepackage{palatino} % Use the Palatino font

\usepackage{microtype} % Improves spacing

\usepackage{multicol} % Required for splitting text into multiple columns

\usepackage[bf,sf,center]{titlesec} % Required for modifying section titles - bold, sans-serif, centered

\usepackage{fancyhdr} % Required for modifying headers and footers
\fancyhead[L]{\textsf{\rightmark}} % Top left header
\fancyhead[R]{\textsf{\leftmark}} % Top right header
\renewcommand{\headrulewidth}{1.4pt} % Rule under the header
\fancyfoot[C]{\textbf{\textsf{\thepage}}} % Bottom center footer
\renewcommand{\footrulewidth}{1.4pt} % Rule under the footer
\pagestyle{fancy} % Use the custom headers and footers throughout the document

\newcommand{\entry}[4]{\markboth{#1}{#1}\textbf{#1}\ {(#2)}\ \textit{#3}\ $\bullet$\ {#4}}  % Defines the command to print each word on the page, \markboth{}{} prints the first word on the page in the top left header and the last word in the top right

%----------------------------------------------------------------------------------------

\begin{document}

%----------------------------------------------------------------------------------------
%	POGLAVJE A
%----------------------------------------------------------------------------------------

\section*{A}

\begin{multicols}{2}

\entry{A1A}{oznaka prekinjanega nosilnega signala}{angl. Continious Wave telegraphy keying the transmitted signal}{oddajanje Morsejevih znakov samo z nosilnim signalom}

\entry{A2A}{oznaka nosilnega signala z dvema bočnima pasovoma}{angl. Telegraphy by the on-off keying of a tone modulated carrier}{oddajanje Morsejevih znakov z obema bočnima pasovoma}

\entry{A3E}{oznaka telefonije z amplitudno modulacijo}{angl. Telephony using amplitude modulation}{telefonija z obema bočnima pasovoma}

\entry{A9W}{oznaka sestavljenega oddajnega signala}{angl. composite emission: double sideband e.g. a combination of telegraphy and telephony}{npr. kombinacija telegrafije in telefonije}

\entry{absolute accuracy}{geodetska ali geografska točnost}{angl. The accuracy of a position with respect to the geographic or geodetic coordinates of the earth (IMO).}{točnost določitve položaja glede na geodetske koordinate na Zemlji (IMO)}

\entry{accuracy}{točnost}{angl. The degree of conformance between the estimated or measured parameter of a craft at a given time and its true parameter at that time (IMO).}{stopnja usklajenosti med ocenjenim in izmerjenim parametrom plovila v določenem času, ki ponazarja resnično vrednost parametra v tem času (IMO).}

\entry{$a_e$}{polmer Zemlje}{angl. Earth’s equatorial radius}{ekvatorialni polmer, tipično $6,378 x 10^6$ m}

\entry{AGC}{način delovanja radia}{angl. Automatic gain control}{samodejno nastavljanje ojačenja oz. dobitka}

\entry{AIS}{sistem obveščanja o istovetnosti ladij}{angl. Automatic Identification System (for shipping - IMO)}{za izmenjavo podatkov ladja-ladja in ladja-obala dinamičnih (vključno poločaj, SOG, COG, hdg, ROT), statičnih (vključno z MMSI, callsign, ime, velikost, vrsta) in potovalnih (vključno draft, cargo, dest, ETA, route plan) podatkih. Uporablja dve VHF frekvenci - 161,97 in 162,025 MHz ter SOTDMA kanali; eno-sekundni okviri so sestavljeni iz 2.250 rež, vsaka vsebuje 256 bitov. Možnost sprejemanja ~ 4.000 poročil/min.}

\entry{ALRS}{seznam}{angl. Admiralty List of Radio Signals}{seznam radijskih signalov v pomorski službi}

\entry{AM}{način modulacije}{angl. Amplitude Modulation}{amplitudna modulacija}

(of Inmarsat III satellites)
 in early 05.

\entry{AOR-E}{območje satelita INMARSAT III}{angl. Atlantic Ocean Region (East)}{vzhodni del Atlantskega oceana, AOR-E na 15,5 $^{\circ}$W}

\entry{AOR-W}{območje satelita INMARSAT III}{angl. Atlantic Ocean Region (West)}{zahodni del Atlantskega oceana, AOR-W na 98 $^{\circ}$W}

\entry{approx}{..}{angl. approximate}{približno}

\entry{ARQ}{način dela v TK}{angl. Automatic Repetition reQuest}{samodejna zahteva za ponovitev (oddajanja), način dela teleprinterja (telex) med dvema stalnima postajama}

\entry{ASCII}{način kodiranja}{angl. American Standard Code for Information Interchange)}{ameriški standarizirani nabor za izmenjavo informacij, 7-bitni osnovni nabor znakov, ki obsega $2^{7} = 128$ znakov; glej \textit{kilobit}}

\entry{ARCC}{posebno središče za SAR}{angl. Associated Rescue Co-ordination Centre)}{središče, ki ga imenuje posebna agencija za SAR, na katerega INMARSAT-ove obalne zemeljske postaje (CES) usmerjajo klice v stiski}

\entry{Autolink RT}{način TK zvez}{angl. automatic communication equipment)}{vsako plovilo, ki ima opremo Autolink RT, lahko vzpostavlja radiotelefonske zveze na ali na VHF ali MF ali HF pasovih, z zemeljskimi radijskimi postajami, ki ravno tako uporabljajo storitev Autolink RT}

\entry{Automatic DSC}{način dela DSC}{angl. special mode of DSC)}{poseben način dela prenašanja digitaliziranih sporočil (DSC), v katerem se uporabljajo samodejno uglasljivi oddajniki, ki brez posredovanja operaterja potrjujejo sprejem (\textit{angl. acknowledgment}) prejetih DSC sporočil na prednastavljenih delovnih frekvencah}

\entry{availability}{razpoložljivost}{angl. the percentage of time that an aid, or system of aids, is performing a required function under stated conditions (IMO).)}{delež časa (v \%), v katerem naprava ali sistem naprav opravlja zahtevano nalogo v predpisanih pogojih (IMO).}

\end{multicols}

%----------------------------------------------------------------------------------------
%	SECTION B
%----------------------------------------------------------------------------------------

\section*{B}

\begin{multicols}{2}

\entry{B9W}{oznaka sestavljenega signala iz analognega in digitalnega dela}{angl. composite emission: independent sidebands with one or more channels containing quantized or digitized information together with one or more channels containing analogue information}{med seboj neodvisni bočni pasovi npr. kombinacija telegrafije in telefonije}

\entry{baud}{b:d, enota za hitrost prenosa podatkov}{angl. a measure for the rate of transfer of binary message (1 bit/s = 1 baud)}{en baud znaša 1 bit na sekundo}

\entry{Beacon identification data}{informacija o istovetnosti klicatelja v stiski}{angl. digitally coded information inside distress beacon and corresponds to the identity of the vessel in distress}{plovilo klicatelja v stiski svojo istovetnost posreduje samodejno}

\entry{bit}{enota za količino podatkov}{angl. a measure for the binary data quantity (\textit{see kilobit})}{količina binarnih podatkov}

\entry{Bn}{radijska postaja}{angl. beacon}{ponavadi oddajna postaja}

\entry{bps}{hitrost prenosa podatkov}{angl. bits per second}{en 1 bit na sekundo}

\entry{brg}{kurz}{angl. bearing}{smer plovbe oz. kot med severom in izbrano smerjo}

\entry{Broadcasting-satellite service}{javna satelitska storitev}{angl. a radiocommunication service in which signals transmitted or retransmitted by space stations are intended for direct reception by the general public}{javna radiokomunikacijska storitev v kateri so signali oddani ali prenešeni uporabnikom preko postaj v vesolju}

\entry{Broadcasting service}{javna radijska storitev}{angl. a radiocommunication service in which transmissions are intended for direct reception by the general public}{javna radiokomunikacijska storitev v kateri so signali namenjeni vsem uporabnikom}

\entry{byte}{količina podatkov}{angl. collection of bits that makes up a binary word}{skupek bitov, ki tvori binarno besedo, vsaj 8 bitov}

\end{multicols}

%----------------------------------------------------------------------------------------
%	SECTION C
%----------------------------------------------------------------------------------------

\section*{C}

\begin{multicols}{2}
		
\entry{$^{\circ}$C}{zapis temperature}{angl. degrees Celsius}{zapis stopinj, ki imajo enoto $1/100$ intervala med temperaturo ledišča in vrelišča vode in ničlo pri ledišču vode}

\entry{CBM}{označevanje boj}{angl. Conventional Buoy Marking}{mednarodni dogovor}
		
\entry{CG}{obalna straža}{angl. Coast Guard}{služba za delo v priobalnem morju posamezne države}

\entry{Ch}{kanal}{angl. Channel}{frekvenčni pas za izbran namen uporabe}

\entry{CNIS}{služba za pomoč pri navigaciji v Kanalu}{angl. The Channel Navigation Information Service helps vessels navigate safely and prevents collisions in the Dover Strait}{služba, ki neprestano oskrbuje z radijskimi in radarskimi informacijami, pomembnimi za navigacijo in preprečevanje trčenj v Rokavskem prelivu}		
		
\entry{Co-ordinator surface search}{pooblaščena ladja za iskanje na morju}{angl. a vessel, other than rescue unit, designated to co-ordinate surface search and rescue operations within a specific search area}{ladja je pooblaščena za koordiniranje iskanja in reševanja, ni pa namenjena samemu reševanju}		

\entry{CES}{vrsta radijske postaje}{angl. Coast Earth Station, also Land Earth Station}{nepremična radijska postaja na kopnem, namenjena za storitve fiksnih satelitskih komunikacij ali včasih tudi nomadskih pomorskih satelitskih komunikacij, zagotavlja radijske zveze pomorskim nomadskim satelitskim storitvam.}

\entry{CIRM}{mednarodni odbor}{fr. Comité International Radio-Maritime}{odbor za radijske zveze v pomorstvu}

\entry{Class A or B}{vrsta naprav AIS}{angl. Class of AIS installation}{vrsta naprav AIS: A za velike ladje, B za plovila za kratkočasne dejavnosti}

\entry{CMG}{kurz med odčitki GNSS}{angl. course made good}{smer po loksodromi med dvema navigacijskima rešitvama položaja satelitskega navigacijskega sprejemnika}

\entry{COG}{kurz med odčitki GNSS}{angl. course over ground, Direction of (the ship’s) movement relative to the Earth, measured on
	board (the ship).}{smer gibanja ladje glede na zemeljski koordinatni sistem, izmerjena na ladji}

\entry{Coast radio station}{kopenska radijska postaja}{angl. a land station in the maritime mobile service}{kopenska postaja, ki omogoča pomorske nomadske komunikacije}

\entry{Coded information}{način prenašanja informacij}{angl. information which is digitally coded within equipmnet by the user to provide information related to distress}{preden se informacija operaterja, povezana s klicem v stiski, prenese naprej, se v radijski postaji digitalno kodira}

\entry{COLREGS}{mednarodni sporazum}{angl. COnvention on internationaL REGulations for preventing collisions at Sea, 1972}{sporazum s pravili kako preprečevati trčenja na morju}

\entry{COSPAS}{satelitski sistem za posredovanje v stiski}{angl. space system for search and distress vessels}{sistem za iskanje plovil in obveščanje v stiski}

\entry{COSPAS-SARSAT system}{satelitski sistem za posredovanje podatkov v stiski}{angl. a satellite-based search and rescue (SAR) distress alert detection and information distribution system, best known for detecting and locating emergency beacons activated by aircraft, ships and backcountry hikers in distress.}{v vozilu, plovilu ali pohodniku v nevarnosti se samodejno aktivira naprava, ki prek satelita obvesti center za posredovanje podatkov iskalcem ponesrečencev}

\entry{CROSS (MRCC)}{regionalno središče za spremljanje in reševanje}{fr. Centres Regionaux de Surveillance et de Sauvetage, angl. Regional centre of operations for surveillance and rescue}
		
\entry{CRS}{vrsta ardijske postaje}{angl. Coast Radio station}{obalna radijska postaja}		
		
\end{multicols}

%----------------------------------------------------------------------------------------
%	SECTION Č
%----------------------------------------------------------------------------------------

\section*{Č}

\begin{multicols}{2}
	
	\entry{Čajka}{vrsta hiperboličnega sistema}{angl. Chayka}{prvi LORAN-C sistem, ki so ga uvedli v Sovjetski zvezi}	

	\entry{čip}{binarni element, digit}{angl. chip}{košček zaporedja v navigacijskih sporočilih, ki sam ne prenaša informacije}	
	
	%	\entry{..}{}{angl.}{..}	 
	
	%	\entry{..}{}{angl.}{..}
	
\end{multicols}	

%----------------------------------------------------------------------------------------
%	SECTION D
%----------------------------------------------------------------------------------------

\section*{D}

\begin{multicols}{2}
	
\entry{dB}{decibel, mera za delež sprejete ali oddane moči}{angl. decibel}{desetkratnik logaritma količnika dveh moči. Če je v količniku pod ulomkovo črto 1 mW, potem se namesto dB uporablja dBm.}
	
\entry{dBW}{decibel vat, normirana decibel, mera za delež moči}{angl. decibel watt}{desetkratnik logaritma količnika neke moči in moči 1 W.}

\entry{DF}{iskanje smeri}{angl. direction finding}

\entry{$D_G$}{razdalja med dvema točkama}{angl. distance between two points on a great circle (orthodromic distance)}{ortodromska razdalja, najkrajša pot med dvema točkama po površini krogle (ne štejemo poti, ki poteka skozi notranjost krogle)}

\entry{DGPS}{diferencialni ameriški satelitski navigacijski sistem}{angl. differential GPS}{način določanja položaja, hitrosti in časa s pomočjo satelitske navigacije in dodatne stalne postaje na Zemlji}

\entry{DGNSS}{diferencialni svetovni satelitski navigacijski sistem}{angl. differential GNSS}{način določanja položaja, hitrosti in časa s pomočjo svetovne satelitske navigacije  in dodatne stalne postaje na Zemlji}	

\entry{DSC}{digitalizirani klic izbranega naslovnika}{angl. digital selective call}{tehnika prenašanja digitaliziranih sporočil, s katero po predpisih CCIR radijska oddajna postaja vzpostavi zvezo in prenese sporočilo drugi postaji ali skupini postaj}	
	
\entry{Distress Alerting}{}{angl. }{}	

\entry{Distress Call}{}{angl. }{}	

\entry{Dsiress Channel}{}{angl. }{}	

\entry{Distress Message}{}{angl. }{}	

\entry{Distress Priority Request Message}{}{angl. }{}	

\entry{DSB}{način v telefoniji}{angl. Dual Side Band}{}	

\entry{DSC}{}{angl. }{}	

\entry{DST}{}{angl. }{}	

\entry{DUT1}{}{angl. }{}	

\entry{DWT}{tonaža ladje}{angl. Dead weight Tonnage}{}	
	
\end{multicols}

%----------------------------------------------------------------------------------------
%	SECTION E
%----------------------------------------------------------------------------------------

\section*{E}

\begin{multicols}{2}
	
	\entry{e.g., example giving}{}{angl.}{na primer}	
	
			
\end{multicols}

%----------------------------------------------------------------------------------------
%	SECTION F
%----------------------------------------------------------------------------------------

\section*{F}

\begin{multicols}{2}

	\entry{firmware}{}{angl.}{programska oprema, ki je prirejena uporabniku oz. izdelani napravi in jo dobite skupaj s strojno opremo v napravi, ki jo kupite}	
	
		
%	\entry{..}{}{angl.}{..} 
	
\end{multicols}
	
%----------------------------------------------------------------------------------------
%	SECTION G
%----------------------------------------------------------------------------------------

\section*{G}

\begin{multicols}{2}
	
	\entry{governor}{}{angl.}{regulator hitrosti vrtenja glavnega ladijskega motorja, nekdaj s centrifugalnim, zdaj večinoma s hidrostatičnim elementom za vzrževanje željene hitrosti}	
	
%	\entry{..}{}{angl.}{..}	 
	
\end{multicols}	

%----------------------------------------------------------------------------------------
%	SECTION H
%----------------------------------------------------------------------------------------

\section*{H}

\begin{multicols}{2}
	
	\entry{hold-in contacts}{}{angl.}{samozadržni kontakti, glej pod zatični ({\em latch}) mehanizem}	
	
%	\entry{..}{}{angl.}{..}	 
	
\end{multicols}	

%----------------------------------------------------------------------------------------
%	SECTION I
%----------------------------------------------------------------------------------------

\section*{I}

\begin{multicols}{2}
	
	
	\entry{interlock}{}{angl.}{varnostni mehanizem, stikalo oz. krmilno vezje s kontaktorji vred, ki preprečuje neželena stanja oz. dogajanja v električnem porabniku; na primer onemogoča nepooblaščen vklop porabnika}
		 	
%	\entry{..}{}{angl.}{..}	 	
	 			
\end{multicols}	

%----------------------------------------------------------------------------------------
%	SECTION L
%----------------------------------------------------------------------------------------

\section*{L}

\begin{multicols}{2}
	
	
	\entry{low-pass filter}{}{angl.}{nizkoprepustni filter, to je filter, ki prepušča nižje frekvennčne komponente signala, visoko frekvenčne pa duši oz. jih ne prepušča - uporablja se v primeru, ko ima naš koristen signal nižjo frekvenco kot motnje}

%	\entry{..}{}{angl.}{..}
		
\end{multicols}	

%----------------------------------------------------------------------------------------
%	SECTION M
%----------------------------------------------------------------------------------------

\section*{M}

\begin{multicols}{2}
	
	
	\entry{motor control centers}{}{angl.}{središča za nadzor krmiljenja motorjev}	
			
\end{multicols}	


%----------------------------------------------------------------------------------------
%	SECTION N
%----------------------------------------------------------------------------------------

\section*{N}

\begin{multicols}{2}
	
	\entry{NC, normally closed}{}{angl.}{mirovni kontakt, v mirovanju sklenjen, rele razklene kontakt, ko ga napajamo}	
	
	\entry{NO, normally opened}{}{angl.}{delovni kontakt, v mirovanju razklenjen (razlaga: rele sklene kontakt, ko ga napajamo)}	 
	
%	\entry{..}{}{angl.}{..}
	
\end{multicols}	

%----------------------------------------------------------------------------------------
%	SECTION O
%----------------------------------------------------------------------------------------

\section*{O}

\begin{multicols}{2}
	
	\entry{overload relay}{}{angl.}{preobremenitveni rele, rele, ki reagira oz. vklopi ali izklopi ob preobremenitvi}
	
%	\entry{..}{}{angl.}{..}	 
	
%	\entry{..}{}{angl.}{..}
	
\end{multicols}	

%----------------------------------------------------------------------------------------
%	SECTION P
%----------------------------------------------------------------------------------------

\section*{P}

\begin{multicols}{2}
	

	\entry{PLC, Programmable Logic Controller}{}{angl.}{PLK, programirljivi logični krmilnik}	 
	
	\entry{PLC-based application}{}{angl.}{rešitev, ki temelji na uporabi PLKja, ko na primer želimo, da se sistem zagona motorja odzove na kombinacijo stanj, ki jih spremljamo preko analognih ali digitalnih vhodov PLK}	 
		
	\entry{PTC -positive temperature coefficient}{}{angl.}{pozitivni temperaturni koeficient}
					
\end{multicols}	

%----------------------------------------------------------------------------------------
%	SECTION R
%----------------------------------------------------------------------------------------

\section*{R}

\begin{multicols}{2}
	
	\entry{reversing DOL starter}{}{angl.}{zaganjalnik z možnostjo izbire smeri zagona električnega motorja}	
	
	\entry{RMS value}{}{angl.}{efektivna vrednost izmenične veličine}
	
	\entry{Read Only Memory}{}{angl.}{ROM, trajni pomnilnik}		
	
	\entry{ROM is non-volatile}{}{angl.}{v ROMu se shranjena stanja ne izbrišejo}	

\end{multicols}	


%----------------------------------------------------------------------------------------
%	SECTION S
%----------------------------------------------------------------------------------------

\section*{S}

\begin{multicols}{2}
	
	
	\entry{solid state electronics}{}{angl.}{polprevodniška elektronika; elementi vezij so narejeni z mikroelektronskimi postopki iz polprevodniških materialov}
	
	\entry{solid state relay}{}{angl.}{polprevodniški rele}	
			
	\entry{strain-gauge base transducer}{}{angl.}{pretvornik z uporovnimi lističi, upori, vezani v wheatstonov mostič so prilepljeni na podlago, ki se upogiba ali zvija, zaradi česar se upornost lističev spreminja, kar spreminja upornost celotnega mostiča - z natančno namestitvijo lističev dosežemo, da je sprememba upornosti sorazmerna velikosti sile, ki je preoblikovala podlago: tlak, vzvoj..}	
			
	%\entry{}{}{angl.}{}	
			
	%\entry{}{}{angl.}{}	

\end{multicols}	


%----------------------------------------------------------------------------------------
%	SECTION T
%----------------------------------------------------------------------------------------

\section*{T}

\begin{multicols}{2}
	
	\entry{threshold value}{}{angl.}{mejna vrednost, prag}	
	
	\entry{thyristor}{}{angl.}{tiristor ( je polprevodniški element za velike električne moči, sestavljen iz največkrat štirih plasti P-N-P-N, ki popolnoma prevaja tok skozenj, če za trenutek prejme krmilni tok, ki vzpostavi napetost na krmilni sponki. Ko s krmilnim tokom napetost s krmilne sponke odvedemo, tok skozi tiristor  preneha. Za razliko od relejev, razen za trenutek, tiristorji ne rabijo krmilnega toka za vzdrževanje odprtosti oz. prevajanja.}	 
	
	\entry{triac-solid state relay}{}{angl.}{polprevodniški rele v izvedbi s triakom}
	
\end{multicols}	

%----------------------------------------------------------------------------------------
%	SECTION V
%----------------------------------------------------------------------------------------

\section*{V}

\begin{multicols}{2}
	
	\entry{voltage built up}{}{angl.}{naraščanje napetosti}	
	
%	\entry{..}{}{angl.}{..}	 
	
%	\entry{..}{}{angl.}{..}
	
\end{multicols}	

%----------------------------------------------------------------------------------------
%	SECTION W
%----------------------------------------------------------------------------------------

\section*{W}

\begin{multicols}{2}	
	
	\entry{wheatstone bridge configuration}{}{angl.}{zasnova wheatstonovega mostiča, wheatstonov mostič sestavljajo štirje upori, vezani tako, da tvorijo obliko kvadrata - po eni diagonali priključimo stabilen enosmerni izvor, po drugi pa z voltmetrom merimo napetost; ko je mostič v ravnotežju, je napetost na voltmetru enaka nič}	
	
%	\entry{..}{}{angl.}{..}	 
	
%	\entry{..}{}{angl.}{..}
	
\end{multicols}	

%----------------------------------------------------------------------------------------
%	SECTION Z
%----------------------------------------------------------------------------------------

\section*{Z}

\begin{multicols}{2}
	
	\entry{zero – correction trimmer}{}{angl.}{nastavljivi upor (trimer) za nastavljanje ničle}	
	
%	\entry{..}{}{angl.}{..}	 
	
%	\entry{..}{}{angl.}{..}
	
\end{multicols}	

%----------------------------------------------------------------------------------------
%	SECTION X
%----------------------------------------------------------------------------------------

\section*{X}

\begin{multicols}{2}
	
%	\entry{..}{}{angl.}{..}	
	
	\entry{Xenon}{ksenon}{angl. xenon}{žlahtni plin}	 
	
%	\entry{..}{}{angl.}{..}
	
\end{multicols}	

	
%------------------------------------------------

\end{document}