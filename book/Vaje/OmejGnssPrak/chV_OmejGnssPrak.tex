%
\chapter{Omejitve GNSS v pomorski praksi}
\label{Vaje:GnssPraks} % Always give a unique label
% use \chaptermark{}
% to alter or adjust the chapter heading in the running head

Spoznali boste nekaj praktičnih težav sprejema GNSS in iz njih izhajajočimi možnimi napačnimi razlagami navigacijskih rešitev. Glede na možnosti in okoliščine boste opazovali in dokumentirali obnašanje sprejemnika v obdobju naravnih in umetno povzročenih motenj.

\section{Pomen naloge}
\label{sec:GnnsPomen}

Zastavili si boste vprašanje: Koliko lahko zaupam svojemu sprejemniku GPS? V katerih pogojih lahko navigacijski rešitvi naprave bolj zaupam, kdaj manj in kdaj sploh ne?

Vsak uporabnik navigacijskega sprejemnika je že opazil, da njegov sprejemnik včasih težje določi svoj položaj. Okoliščine, ki povzročajo večjo negotovost položaja boste našli v literaturi in kot \textit{razbijalci mitov} praktično preverili, če vplivajo na določanje položaja ali ne.  

\section{Vloge v skupini}
\label{sec:Gnss_Vloge}
Skupino sestavljate štirje člani, ki si med seboj tudi pomagate, vendar vsak nosi odgovornost za svoj del.

\begin{table}
	\centering
	\caption{Oris vlog skupine Omejtve GNSS v pomorski praksi}
	\label{tab:GnssVloge} 
	\begin{tabular}{c|c|l}
		\hline\noalign{\bigskip}
		število & odgovornost                 & rezultat dela \\
		\noalign{\smallskip}\hline\noalign{\smallskip}
		1       & iskalec literature          & izvleček najdenega\\
		1       & zapisovalec stanj, napovedi & pregled uresničitev napovedi \\
		1       & priprava in izvedba poskusa & razumevanje poskusa \\
		1       & predstavitev                & izvleček izvirnega dela skupine \\ \hline
		vsi     & poročilo                    & poglobitev razumevanja vesoljskega vremena\\
		\noalign{\smallskip}\hline
	\end{tabular}
\end{table}

\subsection{V pregledu literature se osredotočite in zapišite izvleček}
\label{subsec:GnssPrak_ZapIzvlLit}

\textbf{Nasvet} Bodite praktični. Razmislite kaj boste napisali v izvlečku, predvsem pa najdite odgovor na vprašanje, ki si ga boste kot skupna zastavili v začetku. Na primer: \textit{Kaj podatki o vesoljskem vremenu lahko pomagajo pomorščaku med plovbo po Sredozemlju? Kako jih lahko pridobi in kako upošteva?}


Mednarodne lestvice \textit{nevihtnih pojavov} RSG (angl. Radio blackouts, Solar radiation storms, Geomagnetic storms) boste našli na naslovu (17-5). Trenutne nevarnosti sprememb vesoljskega vremena za radijske zveze najdete na (18-6), za satelitsko navigacijo na (19-7) in za električna omrežja na (20-8).

%Andrej Štern (diapozitivi) 
%\verb|http://www.s50e.si/wp-content/uploads/2012/10/S50E-55LET-S57BAJ.pdf|

%Marko Munih: Prednosti SDR tehnologije na KV in UKV

Za bolj zahtevne bralce priporočamo pregled revije \textit{Space Weather} (5-9).

\subsection{Sestavite pregled kritičnih okoliščin GNSS navigacije}
\label{subsec:GnssPrak_Podat}


\subsection{Potrdite ali ovrzite vpliv kritičnih okoliščin}
\label{subsec:GnssPrak_Posk}
Možne okoliščine...


\subsection{Oprema}
\label{subsec:GnssPrak_Oprema}
Dva različna GNSS sprejemnika..

\section{Viri}
\label{sec:GnssPrak_Viri}

Pregled virov, ki so našteti med opisi nalog članov skupine.



%\paragraph*{ }
%Dostop do navedenih spletnih dokumentov najdete na strani \verb|www.forumgnsss.si|, kjer v rubriki \textit{Za študente} izberite \textit{spletnjača vesoljsko vreme}.


%
% For built-in environments use
%
% Problems or Exercises should be sorted chapterwise
\section*{Naloge}
\addcontentsline{toc}{section}{Problems}
%
% Use the following environment.
% Don't forget to label each problem;
% the label is needed for the solutions' environment
\begin{prob}
	\label{Nal:GnssPrak_Izvl}
	\textbf{Izvleček naj bo uvod v poročilo}\\
	(a) V skupini najdite največ pet preverljivh okoliščin, ki naj bi povečevale negotovost odčitavanja.\\
	(b) Zapišite izvleček, s katerim boste preverjali.
\end{prob}

\begin{prob}
	\label{Nal:GnssPrak_Belez}
	\textbf{Beležite podatke }\\
	(a) Izberite katere podatke o stanju in napovedih boste beležili.\\
	(b) Beležite vsaj 15 dni. \\
	(c) V poročilu kritično zapišite ujemanja med napovedmi in dejanskimi pojavi.
\end{prob}

\begin{prob}
	\label{Nal:GnssPrak_Eksp}
	\textbf{Opazujte vpliv Sončevega vetra na magnetno polje Zemlje}\\
	(a) Sestavite napravo za opazovanje.\\
	(b) Izvedite poskus. \\
	(c) Posvetujte se z ostalimi člani in napišite poročilo o svojih opažanjih.
\end{prob}

\begin{prob}
	\label{Nal:GnssPrak_Predst}
	\textbf{Predstavite opravljeno delo celotne skupine}\\
	(a) Pomagajte ostalim članom pri njihovem delu. \\
	(b) Poskrbite, da je skupina v stiku s predavatelji.\\
	(b) Spremljajte napredek in sestavite osnutek poročila. \\
	(č) Po oddanem poročilu pripravite 5 minutno predstavitev na tabli.
\end{prob}
%
