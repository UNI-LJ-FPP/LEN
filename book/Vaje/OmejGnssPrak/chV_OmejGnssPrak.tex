%
\chapter{Omejitve GNSS v pomorski praksi}
\label{Vaje:GnssPraks} % Always give a unique label
% use \chaptermark{}
% to alter or adjust the chapter heading in the running head

Spoznali boste nekaj praktičnih težav sprejema GNSS in iz njih izhajajoče možne napačne razlage navigacijskih rešitev. Glede na možnosti in okoliščine boste opazovali in dokumentirali obnašanje sprejemnika v obdobju naravnih in umetno povzročenih motenj.

\section{Pomen naloge}
\label{sec:GnnsPomen}

Zastavili si boste vprašanje: Koliko lahko zaupam svojemu sprejemniku GPS? V katerih pogojih lahko navigacijski rešitvi naprave bolj zaupam, kdaj manj in kdaj sploh ne?

Vsak uporabnik navigacijskega sprejemnika je že opazil, da njegov sprejemnik včasih lažje, včasih težje določi svoj položaj. Okoliščine, ki povzročajo povečanje negotovosti položaja, kako sistem sploh deluje, kakšne razsežnosti ima njegova uporaba, kaj .., boste našli v literaturi in kot \textit{razbijalci urbanih legend} praktično preverili, ali v resnici držijo ali ne.  

\section{Vloge v skupini}
\label{sec:Gnss_Vloge}
Skupino sestavljate štirje člani, ki si med seboj tudi pomagate, vendar vsak nosi odgovornost za svoj del.

\begin{table}
	\centering
	\caption{Oris vlog skupine Omejtve GNSS v pomorski praksi}
	\label{tab:GnssVloge} 
	\begin{tabular}{c|c|l}
		\hline\noalign{\bigskip}
		število & odgovornost                 & rezultat dela \\
		\noalign{\smallskip}\hline\noalign{\smallskip}
		2       & iskalec literature          & preizkus legend\\
		2       & priprava in izvedba poskusa & razumevanje poskusa \\
		1       & predstavitev                & izvleček izvirnega dela skupine \\ \hline
		vsi     & poročilo                    & poglobitev razumevanja vesoljskega vremena\\
		\noalign{\smallskip}\hline
	\end{tabular}
\end{table}

\subsection{Literarni preizkus legend}
\label{subsec:GnssPrak_LitPreizkusLegend}

V devetih stavkih imate zapisanih devet urbanih legend, ki so zelo zakoreninjene. S študijem literature jih boste poskusili potrditi ali ovreči.

\begin{enumerate}
	\item GPS je ladijska navigacijska naprava.
	\item GPS zagotavlja samo podatek o položaju.
	\item GPS omogoča uporabo satelitskih slik.
	\item GPS je \textit{veliki brat}, ki dopušča možnost sledenja vsega in vsakogar.
	\item GPS sateliti so geostacionarni, tako kot telekomunikacijski.
	\item Vojaški sprejemniki GPS so bolj natačni kot civilni.
	\item Uporaba GPS je dovoljena, ko plačamo uporabnino.
	\item GPS deluje povsod.
	\item Sistem Galileo bo kmalu deloval.
\end{enumerate}


\textbf{Nasvet} Bodite praktični. Razmislite kaj boste v sklepih napisali o legendah in kako boste z meritvami vsaj tri legende potrdili ali ovrgli.

% NASA video 'ScienceCasts: NASA Spacecraft takes Space GPS to New Heights' iz katerega boste spoznali 'magnetic reconnection' https://www.youtube.com/watch?v=taMzKcehfGw

%tukaj je JRCjeva aplikacija za spremljanje prisotnih wi-fi omrežij, hitrosti prenosov...
% Lahko nastavita kako pogosto bosta oddajali opažanja.
% https://play.google.com/store/apps/details?id=ec.europa.eu.smartmonitor

%Andrej Štern (diapozitivi) 
%\verb|http://www.s50e.si/wp-content/uploads/2012/10/S50E-55LET-S57BAJ.pdf|

%Marko Munih: Prednosti SDR tehnologije na KV in UKV


\subsection{Sestavite pregled kritičnih okoliščin GNSS navigacije}
\label{subsec:GnssPrak_Podat}


\subsection{Potrdite ali ovrzite vpliv kritičnih okoliščin}
\label{subsec:GnssPrak_Posk}
Možne okoliščine...


\subsection{Oprema}
\label{subsec:GnssPrak_Oprema}
Dva različna GNSS sprejemnika..

\section{Viri}
\label{sec:GnssPrak_Viri}

Pregled virov, iz katerih lahko začnete.

\href{http://geology.isu.edu/geostac/Field_Exercise/GPS/GPS_basics_u_blox_en.pdf}{\textit{GPS Basics Introduction to the system Application overview}} \\
\href{http://www.javad.com/downloads/javadgnss/publications/GPS-Tutorial.pdf}{\textit{A GPS Tutorial Basics of High-Precision Global Positioning Systems}} \\
\href{http://spaceplace.nasa.gov/gps-pizza/en/}{\textit{GPS and the Quest for Pizza}} \\
\href{http://www.javad.com/downloads/javadgnss/publications/Javad_Munich_07.pdf}{\textit{The best of GPS, the best of GLONASS and now with: we support Galileo too}} \\
\href{http://www.sappart.net/wp-content/uploads/2014/08/White-Paper_SaPPART_sept15.pdf}{\textit{SaPPART White paper Better use of Global Navigation Satellite Systems for safer and greener transport}} \\
\href{http://www.insidegnss.com/node/4684}{\textit{GNSS Receiver Manufacturers Thwart Jamming, Spoofing Users on ION panel want certification requirement.}}

%
% For built-in environments use
%
% Problems or Exercises should be sorted chapterwise
\section*{Naloge}
\addcontentsline{toc}{section}{Problems}
\begin{prob}
	\label{Nal:GnssPrak_Izvl}
	\textbf{Izvleček naj bo uvod v poročilo}\\
	(a) V skupini najdite največ pet preverljivh okoliščin, ki naj bi povečevale negotovost odčitavanja.\\
	(b) Zapišite izvleček, s katerim boste preverjali.
\end{prob}

\begin{prob}
	\label{Nal:GnssPrak_Belez}
	\textbf{Beležite podatke }\\
	(a) Izberite katere podatke o stanju in napovedih boste beležili.\\
	(b) Beležite vsaj 15 dni. \\
	(c) V poročilu kritično zapišite ujemanja med napovedmi in dejanskimi pojavi.
\end{prob}

\begin{prob}
	\label{Nal:GnssPrak_Eksp}
	\textbf{Izberite tri legende in vsako praktično preizkusite}\\
	(a) Zamislite si tri poskuse.\\
	(b) Izvedite poskuse. \\
	(c) Zabeležite rezultate.
	(č) Posvetujte se z ostalimi člani in napišite poročilo o svojih opažanjih.
\end{prob}

\begin{prob}
	\label{Nal:GnssPrak_Predst}
	\textbf{Predstavite opravljeno delo celotne skupine}\\
	(a) Pomagajte ostalim članom pri njihovem delu. \\
	(b) Poskrbite, da je skupina v stiku s predavatelji.\\
	(b) Spremljajte napredek in sestavite osnutek poročila. \\
	(č) Po oddanem poročilu pripravite 5 minutno predstavitev na tabli.
\end{prob}
%
