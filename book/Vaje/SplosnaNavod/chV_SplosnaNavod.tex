%
\chapter{Splošna navodila za izdelavo seminarske naloge }
\label{ch:NavodSeminar}

\textit{Kaj morate vedeti pred začetkom dela?} Najprej si preberite splošna navodila v tem poglavju. Ko boste vključeni v eno od skupin pa si boste skupaj izbrali eno izmed nalog. Študijske naloge boste opravljali v različnih prostorih: v predavalnici, v navtičnem, v komunikacijskem simulatorju, v laboratoriju za elektrotehniko in na plovilih, ki bodo poleg nujnega vložka predavateljev tudi z vašim prispevkom prirejena za zastavljene naloge.

\textbf{Lastne izkušnje} si boste s seminarskimi nalogami oplemenitili, če jih boste delali premišljeno in boste svoj lastni napredek v dojemanju delili tudi z drugimi v skupini in v letniku. Ne bojte se pravočasno vprašati predavatelje. Oprite se na sistem enomesečnega dela (podrobneje v poglavju \ref{sec:OpisDela}, pomoč\footnote{\textbf{pomoč} razlaga simbolov: o .. samostojno delo v skupini, x .. računajte na pomoč predavateljev}), v katerega kar pogumno vstopite in vztrajajte.
 
\begin{table}[!htbp] 
	\caption{Faze enomesečnega dela s seminarsko nalogo: časovni roki in pričakovani rezultati.}
	\vspace{2mm}
	\begin{center}
		{\small
			\begin{tabular}{l||c|c|l|c}\hline
			faza dela	   & rok za dokončanje & pomoč & rezultati 			     & opombe \\ \hline \hline
			(1) skupine    & 1 teden           & x     & seznam članov           & vprašanja po e-pošti\\ \hline
			(2) moja naloga& 1 teden           & o     & naloge članov sporočim & srečanje skupine\\ \hline
			(3) lab. okolje&                   & (x)   & zapisi, logi, slike     & najava, odložišče \\ \hline
			(4) analiza	   & 1 teden           & x     & analize in načrti    	 & čimbolj samostojno\\ \hline
			(5) del. okolje&                   & x     & zapisi, logi, slike     & vreme, plovila  \\ \hline
			(6) navodila   & 1 teden           & (x)   & navodila                &\textit{za telebane}\\ \hline
			(7) pisanje	   &                   & o     & poročilo                & po \TeX predlogi \\ \hline
			(8) vaših 5 minut& spremljaj objave& o     & predstavitev            & flomaster, tabla \\ \hline
			\textbf{po predstavitvi}&          &       & ocena seminarja         & vklj. medsebojne ocene\\ \hline
				            &                  &       & zbornik poročil         &      \\ \hline
			\end{tabular}
		}
	\end{center}
	\label{tab:faze_dela}
\end{table}

V skoraj vseh fazah dela vam bomo torej na voljo predavatelji in strokovni sodelavci, da boste nalogo dokončali v predvidenem roku.

\section{Enomesečno delo}
\label{sec:OpisDela}

Na prvem predavanju vam bomo predavatelji predstavili načrt dela. Po osnovanju skupin se boste seznanili s potrebno opremo in si razdelili predvidene naloge. Poskuse boste opravljali v predvidenih varnih okoljih laboratorija ali simulatorja, dobljene rezultate boste analizirali po skupinah. Nato boste izvedli poskuse v delovnih okoljih, \textit{ko bo vremenska napoved ugodna, ne odlašajte}. Ko boste poskuse opravili, boste takoj po njih napisali navodila za izvedbo poskusov in sestavili poročila po predlogi (glej poglavje \ref{ch:Poroc_Vzorec}) \ref{Sandro_Predloga}. Delo boste sklenili s petminutno predstavitvijo skupnega dela.     

\textbf{Oceno} seminarske naloge oz. ali jo boste opravili, boste sooblikovali sami. Poleg kakovosti izvedbe, poročila in predstavitve bo ocena vključevala tudi, če ste se držali rokov. 

Če bo katera od nalog od posameznih članov zahtevala bistveno več časa kot 15 ur, boste ob predložitvi dokumentacije in pogovoru s predavatelji lahko oproščeni katere od laboratorijskih vaj. Vsaka skupina pošlje na \textit{kratko usposabljanje} (do pol ure ) enega člana, ki bo odgovoren za obdelavo oz. prikaz rezultatov z grafom. V večini skupin boste vrisovali točke, kurze na svoje pomorske karte.


% Problems or Exercises should be sorted chapterwise
\section*{Naloge}
\addcontentsline{toc}{section}{Naloge}
Če vas zanima ..
\begin{prob}
	\label{Nal:SplNavCas}
	 \textit{Kakšen pomen ima čas v pomorstvu?}, si podrobnosti za izdelavo preberite v poglavju \ref{Vaje:Cas}.
\end{prob}

\begin{prob}
	\label{Nal:SplNavMem}
	\textit{Kaj lahko poveste o poteku plovbe, če dobite v roke časovni graf ali pospeškov ali kotne hitrosti obračanje okoli osi plovila ali zabeleženih magnetnih polj?}, si podrobnosti za izdelavo preberite v poglavju \ref{Vaje:RekonsMems}.
\end{prob}


\begin{prob}
	\label{Nal:SplNavKart}
	\textit{Kako najdete koordinato na morju, izmerite slanost, motnost in globino?}, si podrobnosti za izdelavo preberite v poglavju \ref{Vaj:KartSlanGlob}.
\end{prob}

\begin{prob}
	\label{Nal:SplNavKomp}
	\textit{Kaj morate vedeti o konkretnem kompasu, kako ga pripravite za plovbo in kako plovbo z njim tudi izvedete?}, si podrobnosti za izdelavo preberite v poglavju \ref{Vaje:VrtKompas}.
\end{prob}

\begin{prob}
	\label{Nal:SplNavVes}
	\textit{Kako vesoljsko vreme vpliva na komunikacijo in navigacijo na Zemlji?}, si podrobnosti za izdelavo preberite v poglavju \ref{Vaje:VesVrem}.  
\end{prob}

\begin{prob}
	\label{Nal:SplNavGns}
	\textit{Katere omejitve uporabe podatkov iz sprejemnikov GNSS srečamo v pomorski praksi?}, si podrobnosti za izdelavo preberite v poglavju \ref{Vaje:GnssPraks}.
\end{prob}




%
