%
\chapter{Splošen pregled komunikacijskih ladijskih naprav in osnove GMDSS}
\label{intro} % Always give a unique label
% use \chaptermark{}
% to alter or adjust the chapter heading in the running head

%tukaj je JRCjeva aplikacija za spremljanje prisotnih wi-fi omrežij, hitrosti prenosov...
% Lahko nastavita kako pogosto bosta oddajali opažanja.
% https://play.google.com/store/apps/details?id=ec.europa.eu.smartmonitor
%tukaj pa AKOSova za fiksna omrežja
% http://www.komuniciraj.eu/test-hitrosti#

% https://www.navcen.uscg.gov/?pageName=GMDSS
%In 1979, a group of experts drafted the International Convention on Maritime Search and Rescue, which called for the development of a global search and rescue plan.  This group also passed a resolution calling for the development of a Global Maritime Distress and Safety System (GMDSS) to provide the communication support needed to implement the global search and rescue plan.  

%This system, which the world's maritime nations - including the United States - have implemented, is based upon a combination of satellite and terrestrial radio services and has changed international distress communications from being primarily ship-to-ship-based to primarily ship-to-shore-based (Rescue Coordination Center).  

%For more information, visit the International Maritime Organization (IMO) Global Maritime and Distress Safety Systems Overview and/or download a GMDSS Inspection Checklist. http://transition.fcc.gov/eb/ShipInsp/gmdss_checklist.pdf

%GMDSS: the next generation, As IMO’s review of the GMDSS continues, Iridium is pushing for a place at the table while Inmarsat is looking to evolve its offering to include ‘safety as a service’, Digital Ship October 2015 page 12

%Fairway Issue No. 30 Spring, 2010
%In introducing this meeting on board HQS Wellington Kim Fisher said that he had originally suggested the title as a joke since everything nowadays was e something, eNavigation, eCommerce, but that the joke had returned to haunt him as eGMDSS could become a reality. GMDSS is only ten years old, but the idea goes back much further. Under SOLAS 1960 passenger ships and cargo vessels over 1,600 gt. had to carry radio telegraph and monitor 500 kHz, while smaller vessels carried radio telephone, keeping watch on 2,182 kHz. SOLAS 1974 had introduced VHF radio telephony, keeping watch on channel 16. Watchkeeping was manual, assisted by radio telegraph and radio telephone auto alarms. The range of about 150 nm. under normal conditions was of limited use for ocean passages.
%GMDSS introduced the idea of distress alerts, notified, not directly to ships, but to a shoreside Rescue Co-ordination Centre by terrestrial or satellite communications. The
%RCC would then alert ships in the area and coordinate rescue attempts. This involved dividing the world into Maritime Search and Rescue regions. It was intended to consign morse code, UHF transmissions, 121.5 MHz and manual watchkeeping to history, bringing in their place satellite EPIRBs, SART, NAVTEX, Inmarsat, SafetyNET, DSC and NBDP. Two other changes were brought in at the same time – distress could be applied to persons as well as ships, and the calling party selected the channel for subsequent communications.

%Chris Blockley-Webb from the Navigation Safety Branch of the MCA presented a review of, and modernization of, GMDSS, stating that it was a partnership between
%many administrative bodies, and that it would take twenty years to achieve full implementation. He quoted nine distinct functions of GMDSS.
%Ship to shore distress alerts by two separate and independent means. Shore to ship distress alerts. Ship to ship distress alerts. Search and rescue co-ordination communications.
%On scene communications. Locating signals Maritime safety information. General radio communications. Bridge to bridge communications. These were gone into in some detail. Moving on to the current situation Chris Blockley-Webb described COMSAR 14/7/1 and associated papers, subscribed to by the UK, USA, Chile, France and Australia. With the plans for eNavigation, GMDSS must be part of this, but there is currently no indication of which direction it will take nor how they will be linked. There are likely to be scoping exercises at the next two COMSAR sessions, with review and modernization of the system taking place in 10, 15 and 20 years time. Bob Ball, Electrical Superintendent of BP Shipping, had been booked as the next
%contributor but had been unable to attend at short notice. Instead, Kim Fisher showed a few slides that had been provided. In the first of these it was pointed out that HF radio
%was not well understood by deck officers, that it was more complicated to operate than Sat-C, while having the same costs, but needing greater maintenance. Sat-C, on the other
%hand, was already carried in all sea area A3 vessels for the receipt of SafetyNET, and was carried by most other vessels for LRIT purposes. BP Shipping advocates the dual
%Sat-C approach, with MF / HF fitted purely to meet legislation. They suggest that VHF is still a good solution for sea area A1, being relatively inexpensive and simple to use,
%and that sea areas A2 and A3 be combined and Sat-C carriage be mandatory for this new sea area.
%
%Peter Blackhurst, Head of Maritime Safety Services at Inmarsat, pointed out that technology was changing faster than the regulations and asked if we wanted a slimline, easy operating system that provided the necessary service or an all singing, all dancing system that was swamped with communications protocols. We already had IP Connectivity, SMS and cross-network connectivity, but were being restricted in the use of the radio spectrum. The system could be clogged by communications protocols during distress working, so it was time to look at it again.
%With Inmarsat’s latest I-4 satellites, offering Broadband Global Area Network (BGAN), we now had one device connecting to three networks. Messages could be streamed between the users terminal, voice and ISDN phone or a computer, via the satellite, to an Internet Provider’s router. From there it could go by a Streaming IP network using a guaranteed Quality of Service line or a Standard IP Network by either the internet or a local area network. It could also use a Circuit Switched Network to deliver voice messages to a phone. These systems, and their differences, were described. Safety services supported by Inmarsat were also described, together with possible future developments for both voice and data transmission. Other technologies include the digitization of data for transmission by VHF and the use of WiMax, but all future changes needed to avoid making the earlier analogue systems obsolete. Future eNavigation may eventually use the same communications system as GMDSS.
%The final speaker was Dr. Martin Ziarati , Director of C4FF and a partner in MarEdu Partnership. He pointed out that, as equipment changes and possibly becomes more complex, there is a greater need for training. Citing STWC, which was introduced 15 years ago, research has shown deficiencies, and best practices need to be used worldwide. With the loss of morse code, all mariners are expected to communicate in English, yet that is not the native language of the majority of seafarers. New automatic systems need understanding and emergency situations put people under pressure. Training is required, but how to provide it?
%C4FF is a training and distance learning project that operates through the internet via its web sites. It is supported by several organizations such as BTEC, MNTB, MCA, EDEXCEL and NVQ/SVQ. Where relevant, on-line simulators are used, with over 23,000 users in 190 countries. The simulators are not generic, but are specific to each manufacturer’s equipment. Training is available in ten languages and, being funded by the EU, is free to the user. The web sites through which this is delivered are:
%www.c4ff.co.uk www.egmdss.com www.martel.pro www.maritime-test.org www.maredu.co.uk www.eWoggle.co.uk 

%Following the presentations there were some questions:
%Q: Ships used to carry a Radio Officer, who could not only use the equipment but also repair it when it failed. Who fixes this kit when it is faulty?
%A: There will be a requirement to carry duplicate equipment so that one will carry on working should one become inoperative.
%Q: DSC has proved unreliable. When will it be dumped?
%A: For a time I worked in an MRCC. During that time there were three cases of a person being rescued and families reunited using DSC. The situation is open to discussion.
%Q: Satellite communications are liable to fail. How can this be prevented?
%A: If it happens, GPS satellites are likely to fail as well and the world would be in a mess. It is said that there is a need for alternatives to satellite communication. WiMax is being used successfully in Singapore and digitized VHF experiments for AIS messaging are being carried out in Norway.


Your text goes here. Separate text sections with the standard \LaTeX\
sectioning commands.

\section{Section Heading}
\label{sec:1}
% Always give a unique label
% and use \ref{<label>} for cross-references
% and \cite{<label>} for bibliographic references
% use \sectionmark{}
% to alter or adjust the section heading in the running head
Your text goes here. Use the \LaTeX\ automatism for your citations
\cite{monograph}.

\subsection{Subsection Heading}
\label{sec:2}
Your text goes here.

\begin{equation}
\vec{a}\times\vec{b}=\vec{c}
\end{equation}

\subsubsection{Subsubsection Heading}
Your text goes here. Use the \LaTeX\ automatism for cross-references as
well as for your citations, see Sect.~\ref{sec:1}.

\paragraph{Paragraph Heading} %
Your text goes here.

\subparagraph{Subparagraph Heading.} Your text goes here.%
%
\index{paragraph}
% Use the \index{} command to code your index words
%
% For tables use
%
\begin{table}
\centering
\caption{Please write your table caption here}
\label{tab:1}       % Give a unique label
%
% For LaTeX tables use
%
\begin{tabular}{lll}
\hline\noalign{\smallskip}
first & second & third  \\
\noalign{\smallskip}\hline\noalign{\smallskip}
number & number & number \\
number & number & number \\
\noalign{\smallskip}\hline
\end{tabular}
\end{table}
%
%
% For figures use
%
\begin{figure}
\centering
% Use the relevant command for your figure-insertion program
% to insert the figure file.
% For example, with the option graphics use
\includegraphics[height=4cm]{figure}
%
% If not, use
%\picplace{5cm}{2cm} % Give the correct figure height and width in cm
%
\caption{Please write your figure caption here}
\label{fig:1}       % Give a unique label
\end{figure}
%
% For built-in environments use
%
\begin{theorem}
Theorem text goes here.
\end{theorem}
%
% or
%
\begin{lemma}
Lemma text goes here.
\end{lemma}
%
%
% Problems or Exercises should be sorted chapterwise
\section*{Problems}
\addcontentsline{toc}{section}{Problems}
%
% Use the following environment.
% Don't forget to label each problem;
% the label is needed for the solutions' environment
\begin{prob}
\label{prob1}
The problem\footnote{Footnote} is described here. The
problem is described here. The problem is described here.
\end{prob}

\begin{prob}
\label{prob2}
\textbf{Problem Heading}\\
(a) The first part of the problem is described here.\\
(b) The second part of the problem is described here.
\end{prob}



%
