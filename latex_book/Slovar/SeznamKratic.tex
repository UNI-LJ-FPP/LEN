\chapter*{Seznam kratic}\label{Ch_SeznamKratic} \pagestyle{plain}
      \addcontentsline{toc}{chapter}{Seznam kratic}
% \pagestyle{plain} Tako dose�em, da se glava strani tega poglavja ne izpi�e

Navedene kratice se pojavljajo v besedilu. Nekatere razlage pomenov kratic so podane v besedilu disertacije, ve�ina ostalih pa med pogosto uporabljenimi izrazi v prilo�enem slovarju.

\begin{description}
\item [2dRMS] two distance root mean square
\item [AGPS] Assisted GPS
\item [AHRS] Attitude and Heading Reference System 
\item [AMR] Anisotropic MagnetoResistance
\item [ASIC] Application Specific Integrated Circuit
\item [BOC] Binary Offset Code
\item [BPSK] BiPhase Shift Key
\item [C/A] Coarse Acquisition
\item [CDMA] Code Division Multiple Access
\item [CP] Carrier Phase
\item [DGPS] Differential GPS
\item [DLL] Delay Locked Loop
\item [DMR] digitalni model reliefa
\item [DOP] Dilution of precision 
\item [dRMS] (one) distance root mean square
\item [ECEF] Earth Centered Earth Fixed%, definicija zemeljskega koordinatnega sistema 'mirujo�e Zemlje' 
\item [ECSF] Earth Centered Space Fixed%, definicija astronomskega koordinatnega sistema 'mirujo�ega vesolja' 
\item [EGNOS] European Geostationary Navigation Overlay Service
\item [EKF] Extended Kalman Filter
\item [END] East, North and Down%, definicija koordinatnega sistema  
\item [ESA] European Space Agency
\item [ETRF] European Terrestrial Reference Frame, sestav%, ena od izvedb ITRS
\item [ETRS89] European Terrestrial Reference System 1989%, evropski geodetski koordinatni sistem
\item [EUREF] European Reference Frame
\item [FLL] Frequency Locked Loop
\item [FPGA] Field-Programmable Gate Array
\item [FRD] Forward, Right and Down%, definicija koordinatnega sistema
\item [GAGAN] GPS And Geo Augmented satellite Navigation
\item [GEO] geostationary%, geostacionarni satelit
\item [GIS] geografski informacijski sistemi
\item [GLONASS] GLObalnaja NAigacionaja Sputnikova Sistema 
\item [GLONASS-K] tretja generacija GLONASS (2005-2022)
\item [GLONASS-KM] �etrta generacija GLONASS (2015-2035) 
\item [GNSS] Global Navigation Satellite System
\item [GPS] Global Positioning System
\item [GPR] Ground Penetrating Radar oz. Ground Probing Radar
\item [GRS 80] Global Reference System 1980%, ime globalnega referen�nega sistema
\item [HSGPS] High Sensitivity GPS
\item [ICRS] International Celestial Reference System
\item [IERS] International Earth Rotation Service
\item [IF] Intermediate Frequency
\item [IGS] International GNSS (nekdaj GPS) Service for Geodynamics 
\item [IMO] International Maritime Organisation
\item [IMU] Inertial Measurement Unit
\item [INS] Inertial Navigation System
\item [INS] Integrated Navigation System
\item [ISO] International Organisation for Standardisation  
\item [ITRF] IERS Terrestrial Reference Frame%, sestav ITRF2000 je ena od izvedb ITRS 
\item [ITRS] International Terrestrial Reference System%, ime svetovnega geodetskega sistema
\item [LAAS] Local Area Augmentation System
\item [LAMBDA] Least-squares AMbiguity Decorrelation Adjustment (method)
\item [MEO] Middle Earth Orbit, satelit
\item [MEMS] Micro Electro-Mechanical System
\item [MSAS] Multi-functional transport satellite Satellite-based Augmentation System
\item [NAVSTAR] NAVigation Satellites with Timing And Ranging 
\item [NEU] North, East and Up%, definicija koordinatnega sistema 
\item [NMEA] National Maritime Electronics Association
\item [NSV] Number of Space Vehicles
\item [PDOP] Position Dilution Of Precision
\item [PDR] Pedestrian Dead (Deduced) Reckoning
\item [PLL] Phase Locked Loop
\item [PRN] Pseudo Random Noise
\item [PZ-90] Parametri Zemlji 1990%, nabor parametrov referen�nega sistema za GLONASS
\item [QZSS] Quasi-Zenith Satellite System
\item [RAIM] Receiver Autonomous Integrity Monitoring
\item [RINEX] Receiver independent exchange (format)%, oblika zapisa podatkov GNSS
\item [RMS] root mean square, srednji kvadratni odklon 
\item [RTCM] Radio Technical Commission for Maritime service %, skupno ime sklopa standardov za pomorsko navigacijo in satelitske tehnologije, med drugimi vklju�uje tudi standard za diferencialni GNSS
\item [RTK] Real Time Kinematic
\item [SA] Selective Availability
\item [SBAS] Satellite Based Augmentation System
\item [SM] surface micromachining
\item [ST] satellite tracking %kolikim satelitom sledi sprejemnikov procesor
\item [SNR] Signal to Noise Ratio
\item [SPS] Standard Positioning Service
\item [SV] Space Vehicle
\item [TAI] Temps Atomique International
\item [TCGI] (sistem) Tightly-Coupled GPS INS  
\item [TDOA] Time Difference Of Arrival 
\item [TEC] total electronic content 
\item [TPS] Total Positioning System
\item [TVEC] Total Vertical Electron Content
\item [UERE] User Equivalent Range Error
\item [UTC] Coordinated Universal Time reference 
\item [UT0] Uncorrected Universal Time
\item [UT1] 'popravljeni' Uncorrected Universal Time 
\item [UTM] Universal Transverse Mercator
\item [VLBI] Very Long Baseline Interferometry
\item [VRS] Virtual Reference Station
\item [WAAS] Wide Area Augmentation System %, oznaka ameri�kega satelitskega sistema za zagotavljanje popravkov opazovanj, uveljavila ameri�ko ministrstvo za promet (DoT) in ameri�ka zvezna uprava za letalstvo (FAA)
\item [WADGPS] Wide Area DGPS
\item [WGS84] World Geodetic System 1984 %, ime svetovnega geodetskega sistema, v rabi za GPS  
\end{description}
