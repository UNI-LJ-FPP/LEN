%
\chapter{Preizkus vrtavčnega kompasa in poskusna plovba}
\label{Vaje:VrtKompas} % Always give a unique label
% use \chaptermark{}
% to alter or adjust the chapter heading in the running head

Preučitev statičnih in dinamičnih lastnosti magneto-vrtavčnega kompasa Gyrtotrac ter priprava na poskusno plovbo z njim.

\section{Osnovni napotki}
\label{sec:VrtKompas_OsnNap}
Ponmorščaki že tisočletja iščemo odgovor na vprašanje: \textit{Kje točno je moja ladja?}, zanimata pa nas tudi:
\begin{itemize}
	\item Glede na geografski položaj: \textit{Kje na Zemlji je to?}
	\item Glede na bližnji geografski atribut: \textit{Kaj vidim v svoji bližini?} 	
\end{itemize}

Pomorščaki so si na ta vprašanja odgovarjali z neprestanim opazovanjem smeri s pomočjo ladijskega magnetnega kompasa, ki je prehodil več razvojnih faz.

Sodobna tehnologija elektronske navigacije poskuša izpodriniti klasični tekočinski magnetni kompas in ga nadomestiti z napravami, ki se ne podrejajo zemeljskemu in ladijskemu magnetizmu, imajo pa slabost: odvisne so od električnega napajanja. 

V današnjem času hitrega napredka se ob vsaki navigacijski napravi pojavljajo dodatna vprašanja:
\begin{itemize}
	\item Glede na tehnologijo : \textit{Kakšna naprava je to?},
	\item Glede na uporabo: \textit{Kaj lahko z njo naredim?},
	\item Glede na mejo zaupanja in zanesljivost: \textit{Kolikšno točnost in odstopanja zagotavlja? Kolikšna je za varnost plovbe še sprejemljiva toleranca?},
	\item Glede na rokovanje: \textit{Ali razumem njeno konstrukcijsko sestavo?}, \textit{Ali jo znam zanesljivo uporabljati?}
\end{itemize}

Odgovore na ta in druga vprašanja pomorščaki dobimo v:
\begin{itemize}
	\item nezadostni skromni ustni obliki med prevzemanjem dolžnosti v času primopredaje (zamenjava posadke) in
	\item uporabniškem priročniku, ki ga proizvajalec vedno priloži k napravi.
\end{itemize}

Spoštovani študenti, pred vami je eden od sodobnih segmentov uporabniških elektronskih naprav za vodenje varne navigacije, s katerim se boste morebiti (odvisno od opremljenosti plovila) tudi srečali med plovbo, s katerim boste v bližnji bodočnosti tudi delali: \textbf{kompas GyroTrac}.

\section{Opis naloge}
\label{sec:VrtKompas_Nalog}

\subsection{Opis situacije}
\label{subsec:Nalog_Situac}
Vkrcali ste se v doku na sodobno 68 metrov dolgo jahto, kateri so vam med remontom poleg ostalih sofisticiranih navigacijskih inštrumentov vgradili tudi kompas GyroTrac. Ne pozabite, da ste po končanem prevzemu delovnih dolžnosti glede na delokrog in dolžnosti častnika za vodenje varne navigacijske straže \textbf{kot uporabnik} dolžni \textbf{podrobno} poznati rokovanje z vsemi navigacijskimi inštrumenti na komandnem mostu.

Preden bo jahta dokončno na uporabo predana lastniku, boste v strokovni enote remontne ladjedelnice naredili še poskusno plovbo. Pri tem bodo preverili tudi pravilnost delovanja kompasa Gyro Trac in njegove odzivnosti v različnih pogojih plovbe.

Vsi postopki testiranja kompasa GyroTrac se bodo zapisovali, shranjevali in analizirali, računalniško obdelovali, rezultati zabeležili in na koncu jih boste kot uradne rezultate v črkovni, številčni in grafični obliki predali uporabnikom te naprave v nadaljno uporabo.

Prevzemite položaj poskusne strokovne enote remontne ladjedelnice in izvedite poskus zanesljivost kompasa GyroTrac v vseh pogojih plovbe hitrega plovila na testnem območju Piranskega zaliva.

\section{Izvajanje naloge}
\label{sec:VrtKompas_Izvaj}

\subsection{Sredstva in okolje}
\label{subsec:Izvaj_SredOkol}
\begin{enumerate}
	\item	Področje za izvajanje poskusov = Piranski zaliv (fiksne navigacijske označbe: Rt Bernardin s pripomočki in Rt Madona s pripomočki; pripomočki so fiksne svetlobne oznake na vhodu v lučico)
	\item	Plovilo = šolski m/č Slovenija
	\item	Naprava, čigavo natančnost in zanesljivost preizkušate: = kompas GyroTrac z njegovimi priloženimi komponentami
	\item	Navtični pribor = pomorske karte: Mala karta – 5 in Mala karta – 4 / Izdelava: Geodetski inštitut Slovenije, Ljubljana, 2011.(karte ste dobili skupaj s kompletom orodja za delo na pomorski karti)
	\item	Hitrost plovbe: etapno od 0 do 25 vozlov
	\item	Smer plovbe: naprej, nazaj, kroženje, osmica
	\item	Odklon krmila: etapno od $0^\circ$ do maksimalno preko levega in enako temu preko desnega krila
	\item	Meritve smeri (azimutov/premčnih kotov) v vseh 16 smereh
	\item	Orodje uporabljeno za preizkus: smerna plošča, pomorska karta, navtični pribor, kalkulator
	\item	Ne uporabljajte niti GPS-a, niti magnetnega kompasa, niti drugih pripomočkov in ne primerjamo rezultatov teh naprav z rezultati kompasa GyroTrac – ostale naprave jemljemo posebej - tako poudarimo, da je njihovo delovanje neodvisno od rezultatov, dobljenih s kompasom GyroTrack 
\end{enumerate}

\subsection{Namig}
\label{subsec:Izvaj_Namig}
A) Izdelajte tabelo, ki imela \textit{v glavi dokumenta} te podatke:
\begin{enumerate}
	\item	Plovilo, na katerem se poskusi izvajajo in osnovne karakteristike – glej vpisni list
	\item	Področje in datum izvajanja
	\item	Stanje morja – vzvalovanost, višina valov, smer: uporabi Beaufort-ovo lestvico
	\item	Stanje vremena – uporabi tabelo oblačnosti
	\item	Stanje vetra – smer in moč: uporabi Beaufort-ovo lestvico
	\item	Stanje vidljivosti – uporabi lestvico vidljivosti
    \item   itd, Navedite vse kar ocenjujeta, da bi bilo še potrebno prikazati, npr. temperatura zraka; relativna vlažnost zraka; rubrika za morebitno nenadno spremembo pogojev, … 
\end{enumerate}

\flushleft B) Drugi del tabele pa je razpredelnica:

\begin{enumerate}
	\item	Posnetih zaporednih vzorcev (azimutov/premčnih kotev) v krogu segmenta od $360^\circ$ z napravo, ki jo preizkušate:
	\item	Odčitanih vrednosti s pomorske karte
	\item	Razlika med posnetimi in odčitanimi vzorci
\end{enumerate}

\flushleft C) Tretji del tabele je namenjen grafičnemu prikazu rezultatov v vseh kvadrantih.

\flushleft Č) Četrti del tabele je namenjen vpisu podatkov o osebah, ki so poskus izvedle: ime posameznika in podpis


\section{Gradivo}
\label{sec:VrtKompas_Grad}

 1. \textit{A guide to GyroTrac}, KVH
 \begin{itemize}
 \item Installation Instruction
 \item User’s guide
 \item Technical Manual
 \end{itemize}
 
 2. Poiščite in v poročilu navedite ostala primerna gradiva, ki so vam koristila pri delu.
 
 3. Uporabite navedene pomorske karte in:
 \begin{itemize}
 \item	Določite markantne objekte na obali
 \item	Narišite plan plovbe
 \item	Določite področje / smerne linije na obalo na katerih se bodo izvedla obračanja
 \item	Določite primerne in zanesljive pokrite smeri
 \end{itemize}


\paragraph{\textbf{Preden izplujemo oz. izplujete?}}
\textit{ (bomo šli z njimi, ali bodo šli sami ob spremstvu voditelja čolna) na izvajanje poskusov prosim, da pripravite vse statične elemente s potrebnimi  sekundarnimi pripravami in me s planom in postopki, ki jih sestavljate in načrtujete, seznanite na našem tedenskem oziroma petnajstdnevnem srečanju (kako se dogovorimo).}

%\begin{flushright}
%	Kap. d.pl. Darjan Jagnjič  
%\end{flushright}



%
