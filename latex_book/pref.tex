%
\preface

%% Please write your preface here
%Pač vnesemo določena zlata pravila.

Študentom in tečajnikom, čeprav oboji ljubite prakso, želimo avtorji s knjigo pokazati, da je poznavanje nekaj teorije dobro tudi za človeka, ki želi biti dober praktik.

Namen knjige je kvalitativno opisati elektronske naprave na sodobnih ladjah, večino sklepov podkrepiti tudi s kvantitativnimi opisi njihovega delovanja in uporabe. Z dodanimi nalogami in s prilogo Vaje, izzivamo vsakega bralca, da svoje razumevanje opisov poglobi s praktičnimi preizkusi. Knjiga je razdeljena na dva dela. Prvemu, imenovanemu \textbf{Predavanja}, ki je namenjen študijski poglobitvi na predavanjih podane snovi, sledi še del z naslovom \textbf{Vaje}, ki povzema navodila nalog za utrditev snovi.

Avtorjem knjige se nam zdi potrebno poudariti nasvet, da se študentje lotite poštenega rokodelstva. Vemo, da ste željni vaj, zato boste ob naši spremljavi najprej naredili nekaj praktičnih vaj po lastnem izboru, nato pa boste, upamo, z večjim zanimanjem preštudirali teorijo, ki je v ozadju. Po potrjenih učnih načrtih morate teorijo dojeti in preliti v sposobnost samostojnega reševanja nalog, kakršne vas čakajo v praksi. 

Po resnem študiju in lastnih poskusih znamo teorijo ceniti, ker spoznamo, da neka formula resnično deluje. Ob lastnem študiju nastalo spoznanje vam bo pomagalo razumeti v formule zapisane odnose, da se v množici podatkov ne boste zanašali zgolj na neke občutke nekoga ali na avtoriteto pisca pravil za varno delo, ampak v svoji praksi lahko preverjali delo piscev pravil. Če trdite, da vso teorijo nadomesti praksa, potem smo na začetku knjige še čisto vsak na svojem bregu.

Pomorščak je praktik, skrbi za preživetje v spremenljivih razmerah, v stiski se odziva po občutku. Za svoje preživetje na morju in da je zanesljivo prišel, kamor se je namenil, je od nekdaj uporabljal različne pripomočke. Z vikinškim oz. sončnim kompasom (glej sliko \ref{fig:SoncniKompas}) ali danes z EPIRBom\footnote{Emergency Position-Indicating Radio Beacon}, se možnost pomorščakovega preživetja povečuje, obe napravi sta sad uporabe preverjenih spoznanj. 

Kaj loči praktika od znanstvenika? Oblastniki in trgovci, ki so pomorščake podpirali pretežno ne iz globoke človeške naklonjenosti, so podpirali tudi znanstvenike, da so iznašli kakšen pameten pripomoček za na morje. Z njim so pomorščaki zanesljiveje izpolnili  načrte in imeli več upanja za preživetje, ko jim je šlo za nohte. Znanstveniki so ljudje, ki hočejo premakniti robove spoznanja dejstev človeštva na pošten, čeprav ne vsakomur razumljiv način. In avtorji želimo, da pomemben del svojega časa namenite za poskus doumetja formul, ki jih navajamo v pričujočem študijskem pripomočku.

Narediti novo spoznanje zanesljivo, vredno čimvečjega zaupanja, je mantra vse znanosti. Dober praktik bo znanstvena spoznanja sam preizkušal. Toda kako naredi znanstvenik spoznanje zanesljivo? 

Kako je na primer nemški znanstvenik Georg Simon Ohm prepričal sodobnike, da matematično zapisana zakonitost galvanskih tokokrogov $\vec{J}=\gamma \cdot \vec{E}$ ni le naključje ali plod njegove domišljije? Veljavnost je moral sam preizkusiti in dokazati v vseh možnih okoliščinah, javno objaviti leta 1827, prepričati vse uradne komisije, da se je zakon zares \textit{prijel}. Kasneje se je sicer pokazalo, da je zakonitost leta 1781 opazil že angleški znanstvenik Henry Cavendish, a je ni javno objavil. 

John Harrison pa je bil angleški izumitelj, ki je ob skrbnem razvijanju svojega urarskega rokodelstva spoznal, da je upoštevanje in izkoriščanje znanstvenih odkritij nujno za obstoj njegove obrti. Komisija kraljevih admiralov ga je izzvala, da je izdelal znanstveni instrument, ki je uporaben v praksi. Prisilili so ga v dokazovanje ponovljivosti - ob vsem zadovoljstvu in dobro prikritem navdušenju mu admirali tega izrednega dosežka niso hoteli priznati v razumnem roku. 

Satelitska radijska navigacija (GPS) je plod znanstvenega dela velike skupine. V praksi se je radijska frekvenčna navigacija zaradi udobnosti rabe že tako izkazala, da jo sodobni navigatorji včasih lahkomiselno uporabljajo - pa ne le pomorščaki ...

%\section{Razdelitev knjige}
%\label{sec:Uvd_RazdKnj}
% Always give a unique label
% and use \ref{<label>} for cross-references
% and \cite{<label>} for bibliographic references
% use \sectionmark{}
% to alter or adjust the section heading in the running head


\vspace{1cm}

Res, problem teorije je, če se ne sklada s prakso. Velja tudi, da ko je teorija zaradi prijaznosti do občinstva preveč ohlapna, se ji praksa rada izmuzne. Razlagalec ohlapne teorije postane očitno neverodostojen. Posledica njegove neverodostojnosti ni zgolj, da sam postane tarča posmeha, v najhujšem primeru se v praksi bralcu ali poslušalcu lahko zgodi, da mu ohlapni sklepi povzročajo tudi praktične nevarnosti. Posledic ravnanj po ohlapnih teorijah ne želimo niti sebi, niti vam, ki ste se namenili preštudirati naše gradivo.


%% Please "sign" your preface
\vspace{1cm}
\begin{flushright}\noindent
V Portorožu,\hfill {\it dr. Franc Dimc}\\
\today \hfill {\it kap. d. pl. Darjan Jagnjić}\\
\hfill {\it dr. Aleksander Grm}\\
\end{flushright}


